\documentclass[]{report}
\usepackage{hyperref}
\usepackage{graphicx}
\usepackage{float}
% Title Page
\title{Opening a Coffee place in Munich}
\author{Mauricio Montellano}


\begin{document}
\maketitle

\section*{Introduction}

I always wanted to open a coffee place, but the rent prices in Munich are extremely high. Therefore, I need to select a Borough which satisfies my limitations:


\begin{itemize}
	\item where a coffee would be a popular local,
	\item the rent prices are low,
	\item and the density of population is enough to sustain an additional coffee place.  
\end{itemize}

By choosing a Borough with the lower rent prices, the highest expense in any food local (the rent) is reduced. By choosing the a place where the inhabitants do actually frequent coffee places the risk of the opening is reduced, and therefore the stakeholder that support financially the opening of the coffee place are more willing to give th financial funds for this enterprise.

\section*{Data}
A table with date of the Boroughs in Munich can be found at \url{https://de.wikipedia.org/wiki/Stadtbezirke_M\%C3\%BCnchens}. This table includes name of eac Borough (Stadtbezirk), surface area in km$^2$ (Fl\"ache), inhabitants (Einwohner), density of population per km$^2$ (Dichte) and percentage of foreigners (Ausl\"ander). This table give us information about the density of population which is one of the parameters we need. Additionally we can analyze if the other parameters given are correlated with the success of coffee places in each Borough.

The second point for opening a coffee place is to find the lowest possible rent prices as it usually the highest expense in any food related local. The data related to the rent prices for each Borough of Munich can be obtaine from a well-known real state company of Munich which offers this information in their webpage \url{https://suedbayerische-immobilien.de/Mietpreise-Muenchen-Stadtteile}.

Finally, it is needed to find the geographical location and extension of each borough of Munich. This data can be retrieved for each Borough from a GeoJSON file in \url{https://stekhn.carto.com/tables/munich/public/map}.

In order to find out the most popular venues in each borough Foursquare can be used. It is the particular interest of this work to identify the boroughs where a coffee is between the most popular places. However, since Foursquare is not very popular in Germany the data retrieved might not be as big as expected.

Using all of this Information it should be possible to find the best Borough where a new coffee place could present the highest probability to succeed.

\section*{Methodology}

\subsection*{Extracting data and creating a unique data table}

First we need to clear the data of Munich extracted from \url{https://de.wikipedia.org/wiki/Stadtbezirke_M\%C3\%BCnchens}. The names are translated and some unnecessary columns are deleted.
	
The second step is to obtain the data from the rent prices from	\url{https://suedbayerische-immobilien.de/Mietpreise-Muenchen-Stadtteile}. The data extracted from this webpage need to be treated as some columns values are change when converting the table to a dataframe, this is due to the  use of , instead of . for the decimal separation.
	

Both dataframe obtained from the two webpages are merged into a single one producing a table like the following:	
 \begin{center}
 	\begin{tabular}{ | l | l | l | l | l | l |}
 		\hline
 		
 		\textbf{Boroughs} & \textbf{Surf.}(km$^2$) & \textbf{Hab.} & \textbf{Den.}(per km$^2$)& \textbf{Foreig.} $\%$ & \textbf{Price} EUR/m$^2$ \\ \hline
 		Altstadt-Lehel   &3.15 & 21.454 &  6.820 & 27.4& 15.3 \\ \hline
 		Ludwigsvorstadt-Isar. &4.40 & 54.915 & 12.477  & 32.6& 16.4 \\ \hline    
 		Maxvorstadt     &4.30 & 53.443 &12.435     &     27.9& 15.2  \\ \hline
 		Schwabing-West  &4.36 & 69.407&  15.908     &     23.4& 15.2  \\ \hline
 		Au-Haidhausen   &4.22 & 61.999&  14.693     &     24.7& 15.1 \\ \hline
 		\hline
 	\end{tabular}
\end{center}	
 	\vspace{0.5cm}	

\subsection*{Plotting the Boroughs of Munich}
	
Using the Follium library and the Geojson data file obtained form \url{https://stekhn.carto.com/tables/munich/public/map}. From the Geojson file we obtain the coordinates for each Borough which are afterwards given to Foursquare to retrieve information. A map showing the boroughs of Munich along with the density of population can be obtained, as shown in figure \ref{fig:munich_den_map}.

\begin{figure}[h!]
\centering
\includegraphics[width=1.2\linewidth]{../Documents/Coursera_capstone/Density}
\caption{Density of each Munich Borough}
\label{fig:munich_den_map}
\end{figure}

\subsection*{Retrieving the venues data for each Borough}

We now require a list of the most populat venues for each Borough.It is desired to find the borough in which a coffee place i between the most popular places.  As Foursquare allowS to search around specific coordinates with no option of restricting to a neighbourhood, it was used the average coordinate of all the points in the Geojson file as center of each Borough and a radius of 1 kilometer was used for each one.

	

 
\subsection*{Clustering neighbourhoods} 

The method K means is used for clustering the venues in each Borough. In order to do this the best k has to be chosen. The variation of the score of the fitting respect to the number of clusters can be seen in figure \ref{fig:output_56_2}

\begin{figure}[h!]
\centering
\includegraphics[width=\linewidth]{../Documents/Coursera_capstone/output_56_2}
\caption{score of the klustering respect to the number of of cluster k.}
\label{fig:output_56_2}
\end{figure}

It is noticeable that the maximum score is at k = 2s, but this does not give any relevant information. Two local maximums are seen at k=5 and k=12, the former one is taken for our analysis in order to retrieve more information for our goal. 

\section*{Results}

The clustering of the neighborhoods produce the following results


 \begin{center}
 	\begin{tabular}{ | l | l | l | }
 		\hline
 		
 		\textbf{Cluster} & \textbf{Count} & \textbf{Neighbourhoods} \\ \hline
 		0 & 7 & Schwnth, Altstdt-Lhl, Au-Hdhn, Pasing-Obermzg, Lstdt-Isar, Mxvrstdt, S-West\\ \hline
 		1 & 1 & Aubing-Lochhausen-Langwied\\ \hline
 		2 & 5 & Hadern, Moosach, Laim, Sendling-Westpark, Obergiesing-Fasangarten\\ \hline
 		3 & 2 & Allach-Untermenzing, Thalkirchen-Obersendling-Forstenried-Fürstenried-Solln\\ \hline
 		4 & 1 & Milbertshofen-Am Hart\\ \hline
 		5 & 1 & Neuhausen-Nymphenburg\\ \hline
 		6 & 1 & Trudering-Riem\\ \hline
 		7 & 1 & Ramersdorf-Perlach\\ \hline
 		8 & 1 & Berg am Laim\\ \hline
 		9 & 1 & Bogenhausen\\ \hline
 		10 & 1 & Untergiesing-Harlaching\\ \hline
 		11 & 1 & Schwabing-Freimann\\ \hline

 		\hline
 	\end{tabular}
 \end{center}

The cluster 0 is the one that contains the Boroughs in which a coffee place is between the most popular. In order to see which of this Boroughs satisfies our requirements of low rent prices and enough population, the klusters are plot on top of the heat map for the rent prices and density of the boroughs in Munich, as shown in figure \ref{fig:Density} and \ref{fig:rent}.
\begin{figure}
\centering
\includegraphics[width=1\linewidth]{../Documents/Coursera_capstone/Habitants}
\caption{Habitants per borough and clusters of venues.}
\label{fig:Density}
\end{figure}
\begin{figure}
\centering
\includegraphics[width=1\linewidth]{../Documents/Coursera_capstone/rent}
\caption{Rent prices per borough and clusters of venues.}
\label{fig:rent}
\end{figure}




\section*{Discussion and Conclusion}
From the plots in figure \ref{fig:Density} and \ref{fig:rent} it is seen the isolated red dot (representing cluster 0) at the west side of Munich is a Borough where the rent prices are low, there is a significant amount of habitants and a coffee place is popular. This graphical overlaying of the data mined from different webpages suggest the best  place to open a coffee place with minimum financial risk. It is to be considered that Foursquare is not so popular in Germany, and therefore the data may be insufficient, the analysis done here should be repeated with a more popular data base in the area.    

\end{document}          
